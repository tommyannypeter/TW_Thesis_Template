\documentclass[class=NTHU_thesis, crop=false]{standalone}
\begin{document}


\chapter{Introduction}
\label{chap:Introduction}
In the universe, the galaxies are observed to violate Newton's second law with the rotation of the observable matters which is faster than expected. Thus it is thought that there are invisible matters called Dark Matter (DM) generating additional gravity to accelerate the rotation. The most popular hypothesis is proposed that DM is a stable and electrically natural particle $\chi$ which only has weak and gravitational interaction with the Standard Model (SM) particles. Thanks to the characteristics, we can try to search such particle at the Large Hadron Collider (LHC).

One possible model of DM is a Type-II two-Higgs-doublet model (2HDM) with an additional U(1)$_{Z^\prime}$ gauge symmetry, known as the $Z^\prime$-2HDM model \cite{Berlin2014}. A light scalar $h$, which is identified as the SM Higgs boson, and a pseudo-scalar $A$ are introduced among this model, together with a striking process shown in \Cref{fig:DM-Model}. In the $Z^\prime$-2HDM model, the doublets $\Phi_u$ and $\Phi_d$ couple to up-type quarks and down-type quarks and leptons, respectively:
\begin{equation}
-\mathcal{L} \supset y_uQ\widetilde{\Phi}_u\bar{u} + y_dQ\Phi_d\bar{d} + y_eL\Phi_d\bar{e} + \mathrm{h.c.}.
\end{equation}
The doublets attain vacuum expectation values $v_u$ and $v_d$ after electroweak symmetry breaking. In unitary gauge, the doublets are parametrized as
\begin{equation}
\begin{cases}
\Phi_d = \frac{1}{\sqrt{2}}\begin{pmatrix}-\sin\beta H^+ \\ v_d - \sin\alpha h + \cos\alpha H - i\sin\beta A \end{pmatrix} \\
\Phi_u = \frac{1}{\sqrt{2}}\begin{pmatrix}\cos\beta H^+ \\ v_u + \cos\alpha h + \sin\alpha H + i\cos\beta A \end{pmatrix}
\end{cases}
\end{equation}
where $h$ and $H$ are neutral scalars, $A$ is a neutral pseudo-scalar and $\alpha$ is the mixing angle which diagonalizes the $h-H$ mass squared matrix. In addition, $\tan\beta$ is defined to be $v_u/v_d$. The light scalar $h$ is assumed to correspond to the SM Higgs boson with $m_h \sim 125\, \mathrm{GeV}$. The remaining scalars, $H$, $A$ and $H^\pm$ are assumed to have masses above $300\, \mathrm{GeV}$, according to the $b \to s\gamma$ constraint \cite{BRANCO20121}. Furthermore, $\alpha = \beta - \pi/2$ is taken as the alignment limit where $h$ has SM-like coupling, and $\tan\beta \ge 0.3$ is used to keep the perturbativity of the top-quark Yukawa coupling. In total, there are five parameters in the $Z^\prime$-2HDM model, including the masses of the involved particles, $m_A$, $m_{Z^\prime}$ and $m_\chi$, as well as the gauge coupling of the $Z^\prime$, $g_Z$ and $\tan\beta$.

\fig[0.45][fig:DM-Model][!hbt]{DM-Model.png}[The aimed process of the production of DM $\chi$ in the $Z^\prime$-2HDM model. A SM Higgs boson decaying to a pair of $b$-quarks is produced through a $Z^\prime$ mediator coupled to a pseudo-scalar Higgs boson $A$, which eventually decays to undetectable $\chi\bar{\chi}$.]

The process has the attribute targeted by the collider searches that the final state is DM following with a detectable particle, the SM Higgs boson $h$ in this case. Due to the single visible Higgs boson in the final state, the search for DM of $Z^\prime$-2HDM model is called the monoH analysis. This analysis exploits the largest decay branching ratio of a Higgs boson into a pair of $b$-quarks. The signature of the DM model is large transverse missing energy with two $b$-quarks. However, while the transverse missing energy goes larger, these two $b$-quarks go closer to each other in topology. Thus the the resolved region and the merged region are defined for lower and higher missing transverse energy respectively. These two regions use different object reconstruction and event selections, described in \Cref{chap:estimation_strategy}.

The physical object reconstruction used in the analysis is described in \Cref{chap:object_reconstruction}, and the event selection is described in \Cref{chap:event_selection}. Being the biggest background, the $Z$($\nu\nu$) + jets background is estimated by the two-lepton CR. The shape uncertainty of the estimation is described in \Cref{chap:Z_bkg_estimation}. At last, the dominant uncertainties are detailed in \Cref{chap:systematic_uncertainties}.

The analysis separate the transverse momentum into four bins, and uses the binned likelihood approach to fit. Besides the increased luminosity, there are some new features significantly improving the results. The results are eventually interpreted as exclusion limits at 95\% confidence level and documented in \Cref{chap:result}.

\end{document}