\documentclass[class=NTHU_thesis, crop=false]{standalone}
\begin{document}

\chapter{The ATLAS Detector}
\label{chap:ATLAS_Detector}
The ATLAS (A Toroidal LHC ApparatuS)\cite{1748-0221-3-08-S08003} is a multifunctional detector with a nominally forward-backward symmetric cylindrical geometry with respect to the interaction point. It comprises four detector components which are the inner detector tracking system, the electromagnetic (EM) calorimeter, the hadronic calorimeter and the muon spectrometer. The overview of the ATLAS is shown in \cref{fig:ATLAS}.

\fig[0.7][fig:ATLAS][!hbt]{ATLAS.jpg}[The overview of the ATLAS.]

The inner detector is composed of three sub-detectors, the silicon pixel layers, the silicon microstrip (SCT) layers and the transition radiation tracker (TRT), shown in \cref{fig:ID}. The silicon pixels and the SCT are placed in the range of $\left|\eta\right| < 2.5$, providing the information of pattern recognition, outstanding momentum resolution and both primary and secondary vertex measurements. The TRT comprises many layers of gaseous straw-tube tracking detectors in the range of $\left|\eta\right| < 2.0$, taking charge of continuous charged-particle tracks to improve the pattern recognition and enhance the momentum resolution and also giving the electron identification.

\fig[0.7][fig:ID][!hbt]{ID.jpg}[The schematic diagram of the ATLAS inner detector, which comprises three sub-components, pixels, SCT and TRT. Besides the three pixel layers shown in the figure, there is also a newly added layer, the Insertable B-Layer (IBL)\cite{Capeans:1291633}\cite{CERN-LHCC-2012-009}, as the innermost silicon pixel, improving the identification of displaced vertices and the $b$-tagging performance.]

The calorimeters measure electron, photon, jet and $\tau$ lepton energies, covering the range of $\left|\eta\right| < 4.9$, shown in \cref{fig:Calorimeter}. The EM calorimeters are liquid argon (LAr) detectors which are accommodated in three cryostats, two end-caps and one barrel, covering the range of $\left|\eta\right| < 3.2$. The LAr is used because of its inherent behavior, stability of response over time and intrinsic radiation-hardness.

Outside the EM calorimeter envelope, the hadronic calorimeter can be divided into the LAr hadronic end-cap calorimeters, the LAr forward calorimeters and the tile calorimeters for the range of $\left|\eta\right| < 4.9$. The LAr hadronic end-cap calorimeters cover the range of $1.5 < \left|\eta\right| < 3.2$, sharing the cryostats with the EM end-cap calorimeters and the forward calorimeters. The LAr forward calorimeters are placed in the range of $3.1 < \left|\eta\right| < 4.9$. As mentioned previously, the LAr is chosen for its inherent excellent property. The tile calorimeters consist of one central barrel calorimeter and two extended barrel calorimeters, made of steel as the absorber and scintillating tiles as the active material, laying over the range of $\left|\eta\right| < 1.7$.

\fig[0.8][fig:Calorimeter][!hbt]{Calorimeter.jpg}[The schematic diagram of the calorimeters, composed of the EM calorimeter and the hadronic calorimeter.]

The outmost component is the muon spectrometer. Over the range of $\left|\eta\right| < 2.7$, the high-precision tracking chambers measure muon tracks in the large superconducting air-core toroid magnets. There are also the trigger chambers covering the range of $\left|\eta\right| < 2.4$ for providing bunch-crossing identification, providing well-defined $p_T$ thresholds, and measuring the muon coordinate in the direction orthogonal to that determined by the precision-tracking system.

The superconducting magnetic system\cite{Magnet_System} is significantly featured in the ATLAS. As shown in \cref{fig:ATLAS}, the magnet system consists of one solenoid magnet, one barrel toroid magnet and two end-cap toroid magnets. The solenoid magnet gives $2 T$ magnetic field for the inner detector, bending the tracks of charged particles for the momentum measurement and minimizing the radiative density in front of the barrel EM calorimeter. The barrel toroid magnet and the end-cap toroid magnets produce a $0.5 T$ and $1 T$ toroidal magnetic field in the central and end-cap regions respectively, supplying the bending power for the muon detectors.

Apart from the detector structure, there are also the trigger system\cite{Trigger_System} carrying out the event selection. There are up to one billion proton-proton collisions per second in the ATLAS and the trigger system only selects 100 events worth retaining per second. There are three levels for the selection process. The Level-1 trigger searches for high $p_T$ muons, electrons, photons, jets, $\tau$ leptons and large missing and total transverse energy. The Level-2 trigger tags these events with some specific regions of interest, leaving below 3500 events per second. In the end, approximately 200 events after the Level-3 trigger.
\end{document}