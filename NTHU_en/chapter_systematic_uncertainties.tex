\documentclass[class=NTHU_thesis, crop=false]{standalone}
\begin{document}

\chapter{Systematic Uncertainties}
\label{chap:systematic_uncertainties}
In the analysis, the systematic uncertainties come from the reconstruction of physics objects and from theoretical predictions of both the signals and backgrounds. The dominant experimental systematic uncertainties originate from the $b$-tagging efficiency, which is mainly from the calibration of the flavor tagging efficiency of $t\bar{t}$ events \cite{ATL-PHYS-PUB-2017-013}. The experimental systematic uncertainties from the reconstruction of the $E^{miss}_T$, the leptons and the jets are obtained with the tools provided by the E/gamma, MCP and JetEtMiss combined performance group.

The theoretical systematic uncertainties are sourced from the modeling of the signal and the background processes, the MC statistics and the luminosity. The uncertainty of the modeling is from the processes of $V$ + jets, $t\bar{t}$, SM Higgs boson decaying to $b\bar{b}$ associated with a vector boson ($VH(bb)$) and diboson for the most part, which is derived in the SM $VH(bb)$ analysis \cite{Aaboud2017}. The estimate of the systematic uncertainties of the $V$ + jets, the biggest backgrounds, depend on two studies. One is varying the parameters of different MC generators to evaluate the acceptance uncertainties. The other is using data-driven shape comparisons in high purity regions, as described in \Cref{chap:Z_bkg_estimation}. Besides, the luminosity uncertainty is obtained from calibrations of the luminosity scale \cite{Aaboud2016}. The ranking of the dominant systematic uncertainties is shown in \Cref{table:uncertainty_ranking}, with the statistical uncertainty and the total uncertainty of the analysis.

\begin{table}[h]
	\centering
	\begin{tabular}{ l c c c }
		\hline
		\multirow{2}{*}{Source of uncertainty} & \multicolumn{3}{c}{Impact (\%)} \\ \cline{2-4}
		 & (a) & (b) & (c) \\ \hline
		$b$-tagging & 4.0 & 8.0 & 10 \\
		$V$ + jets modeling & 3.5 & 6.0 & 5.0 \\
		Top modeling & 3.7 & 4.8 & 4.5 \\
		MC statistics & 1.8 & 5.4 & 4.9 \\
		SM $Vh$($b\bar{b})$ & 0.8 & 3.2 & 2.1 \\
		Diboson modeling & 0.8 & 1.5 & 1.1 \\
		Signal modeling & 3.0 & 2.5 & 1.5 \\
		Luminosity & 2.0 & 2.5 & 2.5 \\
		Small-$R$ jets & 1.4 & 3.0 & 2.0 \\
		Large-$R$ jets & 0.2 & 1.0 & 2.0 \\
		$E^{miss}_T$ & 1.2 & 1.7 & 1.1 \\
		Leptons & 0.2 & 0.8 & 0.7 \\ \hline
		Total systematic uncertainties & 6.5 & 13 & 13 \\
		Statistical uncertainty & 2.3 & 20 & 23 \\
		Total uncertainty & 7 & 24 & 25 \\ \hline
	\end{tabular}
	\caption{The dominant source of the systematic uncertainties after the fit. Three representative sets of the parameters of $Z^\prime$-2HDM are shown: (a) $(m_{Z^\prime}, m_A) = (0.6\, \mathrm{TeV}, 0.3\, \mathrm{TeV})$, (b) $(m_{Z^\prime}, m_A) = (1.4\, \mathrm{TeV}, 0.6\, \mathrm{TeV})$ and (c) $(m_{Z^\prime}, m_A) = (2.6\, \mathrm{TeV}, 0.3\, \mathrm{TeV})$, assuming the total cross-sections of (a) $452\, \mathrm{fb}$, (b) $3.75\, \mathrm{fb}$ and (c) $2.03\, \mathrm{fb}$. The total uncertainty is the quadrature sum of the statistical and the total systematic uncertainties.}
	\label{table:uncertainty_ranking}
\end{table}

\end{document}