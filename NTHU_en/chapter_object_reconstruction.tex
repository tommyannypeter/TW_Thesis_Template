\documentclass[class=NTHU_thesis, crop=false]{standalone}
\begin{document}

\chapter{Object Reconstruction and Selection}
\label{chap:object_reconstruction}
\section{Jets}
Jets are a method to describe hadronic showers in detectors, consisting of multiple objects. Jets are reconstructed with the anti-$k_t$ algorithm \cite{1126-6708-2008-04-063}. Depending on the toplogical clusters, the radius parameter of $R = 0.4$ is the small-radius (small-$R$) jets and of $R = 1.0$ is the large-radius (large-$R$) jets.

The small-$R$ jets with $p_T > 20\, \mathrm{GeV}$ for $\left|\eta\right| < 2.5$ are regarded to as the central jets. The MV2c10 discriminant \cite{ATL-PHYS-PUB-2015-022} is used for $b$-tagging the central jets to identify the $b$-hadrons, with the 77\% efficiency. The efficiency is defined in the MC simulation as the ratio of the number of the $b$-tagged jets to the number of all jets with the truth value of $b$-jet. In addition, the small-$R$ jets with $p_T > 30\, \mathrm{GeV}$ for $2.5 < \left|\eta\right| < 4.5$ are the forward jets. The large-$R$ jets are required to have $p_T > 200\, \mathrm{GeV}$ and $\left|\eta\right| < 2.0$. The reconstructed topologies of the large-$R$ jets cover multiple constituent jets which are called the track jets. The track jets are linked to the large-$R$ jets with ghost association method \cite{1126-6708-2008-04-005} \cite{CACCIARI2008119}. The flavor identification for the large-$R$ jets depend on the ghost-associated track jets with the MV2c10 discriminant with the 77\% $b$-tagging efficiency. The track jets are described more in the following.

The fixed-radius (FR) track jets are applied for $b$-tagging in the previous analysis, using the anti-$k_t$ algorithm with fixed radius parameters. Instead of the FR track jets, the variable-radius (VR) track jets \cite{0903.0392} are implemented in the analysis for improving the $b$-tagging efficiency. The primary characteristic of the VR track jets is the relevance between the $p_T$ and the jet radius parameter: 
\begin{equation}
R \to R_{eff}(p_T) \approx \frac{\rho}{p_T}
\end{equation}
where $\rho$ is a constant factor.

\section{Leptons}
Electrons are reconstructed with energy deposits in the EM calorimeter which matches tracks in the ID. Besides the basic reconstruction, there are two types of electrons for the further selection in the analysis, the baseline electrons and the signal electrons. The baseline electrons have a loose criterion with $\left|\eta\right| < 2.47$ and $p_T > 7\, \mathrm{GeV}$, while the signal electrons require $\left|\eta\right| < 2.47$ and $p_T > 27\, \mathrm{GeV}$. The signal electrons are used to be tighter requirement for the regions requesting events to have electrons. The baseline electrons are used for other requirement about electrons, like electron vetoes.

The muon reconstruction relies on the muon spectrometer in the range of $\left|\eta\right| < 2.7$. It's also claimed to match the tracks in the ID for $\left|\eta\right| < 2.5$. Like electrons, there are two categories of muons, the baseline muons and the signal muons. The baseline muons is required to be with $\left|\eta\right| < 2.7$ and $p_T > 7\, \mathrm{GeV}$. The signal muons are defined with the criterion of $\left|\eta\right| < 2.5$ and $p_T > 25\, \mathrm{GeV}$. Same as the electrons, the signal muons are used to be the criterion for the region which needs events to hold muons, while the baseline muons are for other requirement conditions about muons.

Because of the hadronic-like nature of taus, the reconstruction is complicated and described in \cite{ATLAS-CONF-2017-029}. For the tau vetoes in the analysis, the taus are required to have one or three tracks and have $\left|\eta\right| < 2.5$ and $p_T > 20\, \mathrm{GeV}$, excluding the crack region $1.37 < \left|\eta\right| < 1.52$.

\section{Missing Transverse Momentum}
The missing transverse momentum is determined as the negative vector sum of the transverse momentum of all the reconstructed and calibrated objects in an event. Additionally, the track-based soft term is added into the missing transverse momentum in the analysis, which is from the tracks that not associated with any reconstructed object but still linked to the primary vertex. The absolute value of the missing transverse momentum is expressed as $E^{miss}_T$. Also, there is a purely track-based missing transverse momentum calculated as the negative vector sum of the transverse momentum of all the reconstructed tracks associated with the primary vertex. The absolute value of the track-based missing transverse momentum is denoted as $p^{miss}_T$.

Furthermore, to recognize whether the $E^{miss}_T$ is from weakly interacting particles or from the mis-measurement of multi-jet processes, two kinds of the $E^{miss}_T$ significance are applied. One is the event-based $E^{miss}_T$ significance defined as $E^{miss}_T/\sqrt{{\sum}p^{lepton}_T+{\sum}p^{jet}_T}$ and performed in the previous monoH analysis \cite{Meehan:2225941}, and the other is the object-based $E^{miss}_T$ significance which is newly introduced in the analysis, in order to reject multi-jet backgrounds. The object-based $E^{miss}_T$ significance is defined in \cite{ATLAS-CONF-2018-038}.

\end{document}