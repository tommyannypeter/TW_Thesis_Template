\documentclass[class=NTHU_thesis, crop=false]{standalone}
\begin{document}

\chapter{Results}
\label{chap:result}
The binned likelihood approach is used to get the final fitting result. The data are binned into four regions in $E^{miss}_T$ ($p^V_T$ in CR) to fit simultaneously, three for the resolved region, $[150\, \mathrm{GeV}, 200\, \mathrm{GeV})$, $[200\, \mathrm{GeV}, 350\, \mathrm{GeV})$ and $[350\, \mathrm{GeV}, 500\, \mathrm{GeV})$, and one for the merged region, $[500\, \mathrm{GeV}, \infty)$. For the SR, the mass of the Higgs boson candidate $m_h$ is the fit variable. For the background estimation, the charge of $\mu$ is used as the discriminating variable in one-muon CR to separate $W$($\mu\nu$) + jets and $t\bar{t}$. The $t\bar{t}$ is expected to have equivalent $\mu^+$ and $\mu^-$, while the $W$ + jets process is expected to have more $\mu^+$ than $\mu^-$. In the two-lepton CR, the total yield is used as the fit variable. The jets containing $b$- or $c$-quarks, expressed as heavy flavor jets (HF), dominate in the main background. The normalization of $Z$ + HF, $W$ + HF and $t\bar{t}$ is free parameters in the fit and the different flavor composition between the two jets, like $bb$, $bc$, $bl$ ($l$ for light quarks) and $cc$ provides the systematic uncertainties. The normalization factors of $Z$ + HF, $W$ + HF and $t\bar{t}$ are fitted to be $1.42 \pm 0.10$, $1.51 \pm 0.22$ and $1.10 \pm 0.08$, respectively. For the sub-dominant background, the MC simulation is constrained by theory prediction.

Thanks to the application of the VR track jets, the object-based $E^{miss}_T$ significance, as well as the increased luminosity, the result of the analysis becomes further improved, as shown in \Cref{fig:VRvsFR}. The distribution of $E^{miss}_T$ for the SR, including the resolved region and merged region combined, is shown in \Cref{fig:result-MET-mass} and no significant excess is observed. The results are further interpreted as exclusion limits at 95\% confidence level (CL). The exclusion contour of the $Z^\prime$-2HDM parameters $(m_{Z^\prime}, m_A)$ is depicted in \Cref{fig:result-limit} and the observation is consistent with the expectation.

\fig[0.65][fig:VRvsFR][!hbt]{VRvsFR.pdf}[The comparison of the expected upper limit on the signal strength between the analysis with VR track jets and the previous iteration of the analysis with FR track jets. Other differences between the two analyses also include the application of the object-based $E^{miss}_T$. Obvious improvement is shown in the high $m_{Z^\prime}$ region.]

\fig[0.65][fig:result-MET-mass][!hbt]{result-MET-mass.pdf}[The distribution of $E^{miss}_T$ for the SR, including the resolved region and merged region combined. The solid histogram and the blue dashed line in the upper panel present the comparison between the data and the SM prediction after and before the fit, respectively. The expected $Z^\prime$-2HDM signal is also shown by the red dashed line. The lower panel shows the ratio of the data to the post-fit SM prediction. No significant excess is observed.]

\fig[0.65][fig:result-limit][!hbt]{result-limit.pdf}[The exclusion contour of the $Z^\prime$-2HDM parameters $(m_{Z^\prime}, m_A)$ for other fixed parameter value, $tan(\beta) = 1$, $g_Z = 0.8$ and $m_\chi = 100\, \mathrm{GeV}$. The observation is consistent with the expectation within the uncertainty. The previous ATLAS result is also shown.]

\end{document}