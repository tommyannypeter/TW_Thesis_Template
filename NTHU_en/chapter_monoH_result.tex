\documentclass[class=NTHU_thesis, crop=false]{standalone}
\begin{document}

\chapter{Review on the MonoH Analysis}
\section{Dark Matter and the $Z^\prime$-2HDM Model}
In the universe, the galaxies are observed to violate Newton's second law with the rotation of the observable matters which is faster than expected. Thus it is thought that there are invisible matters called Dark Matter (DM) generating additional gravity to accelerate the rotation. The most popular hypothesis is proposed that DM is a stable and electrically natural particle $\chi$ which only has weak and gravitational interaction with the Standard Model (SM) particles. Thanks to the characteristics, we can try to search such particle at the Large Hadron Collider (LHC).

One possible model of DM is a Type-II two-Higgs-doublet model (2HDM) with an additional U(1)$_{Z^\prime}$ gauge symmetry, known as the $Z^\prime$-2HDM model. A light scalar $h$, which is identified as the SM Higgs boson, and a pseudo-scalar $A$ are introduced among this model, together with a striking process shown in Figure 3.1. The process has the attribute targeted by the collider searches that the final state is DM following with a detectable particle, the SM Higgs boson $h$ in this case. Due to the single visible Higgs boson in the final state, the search for DM of $Z^\prime$-2HDM model is called the monoH analysis. This analysis exploits the largest decay branching ratio of a Higgs boson into a pair of b-quarks. The following sections will briefly present the result of the analysis.

\fig[0.45][!hbt]{DM-Model.png}[The aimed process of the production of DM $\chi$ in the $Z^\prime$-2HDM model. A SM Higgs boson decaying to a pair of $b$ quarks is produced through a $Z^\prime$ mediator coupled to a pseudo-scalar Higgs boson $A$, which eventually decays to undetectable $\chi\bar{\chi}$.]

\section{Background Estimation Strategy}
The main background processes of the analysis are $Z$($\nu\nu$) + jets, $W$($l\nu$) + jets and $t\bar{t}$, shown in Figure 3.2. In the $Z$($\nu\nu$) + jets process, the undetectable neutrinos in the final state are regarded as the missing energy. Thus once the jets are tagged as $b$-jets, the case will be background. In the $W$($l\nu$) + jets process, once the lepton is misidentified as the missing energy and the jets are tagged as $b$-jets as mentioned previously, the procedure will be interference as well. In the $t\bar{t}$ case, if both the leptons are misidentified, the process will be background.

\begin{figure}[!hbt]
	\captionsetup[subfigure]{labelformat=empty}
	\centering
	\subcaptionbox
	{$Z$($\nu\nu$) + jets
		\label{fig:subfig_fig1}}
	{\includegraphics[width=0.3\linewidth]{Zvv.png}}
	~
	\subcaptionbox
	{$W$($l\nu$) + jets
		\label{fig:subfig_fig2}}
	{\includegraphics[width=0.3\linewidth]{Wlv.png}}
	~
	\subcaptionbox
	{$t\bar{t}$
		\label{fig:subfig_fig3}}
	{\includegraphics[width=0.3\linewidth]{ttbar.png}}
	\caption{The main background processes of the monoH analysis.}
	\label{fig:label}
\end{figure}

To constrain these background, the control region (CR) strategy is used. Before defining the CR, the signal region (SR) is defined as where the signals are expected to show. The SR is required to have no lepton based on the model. Then the CR is defined as the orthogonal region to the SR. Exploiting the similar kinematics, the CRs use some better-measured processes in order to constrain the background. There are two CRs in this analysis, the 1-lepton CR and the 2-lepton CR. The 1-lepton CR contains two processes, $W$($l\nu$) + jets and $t\bar{t}$, constraining the same processes in the SR. It requires different number of misidentified lepton, for $W$($l\nu$) + jets, one misidentified lepton in the SR but none in the 1-lepton CR, and for $t\bar{t}$, two misidentified leptons in the SR but one in the 1-lepton CR. The 2-lepton CR exploits the $Z$($ll$) + jets, constraining the $Z$($\nu\nu$) + jets process.

\section{Object Reconstruction}
\subsection{Jets}
Jets are a method to describe hadronic showers in detectors, consisting of multiple objects. Jets are reconstructed with the anti-$k_t$ algorithm. Depending on the toplogical clusters, the radius parameter of $R = 0.4$ is the small-radius (small-$R$) jets and of $R = 1.0$ is the large-radius (large-$R$) jets.

The small-$R$ jets with $p_T > 20 GeV$ for $\left|\eta\right| < 2.5$ are regarded to as the central jets. The MV2c10 discriminant is used for $b$-tagging the central jets to identify the $b$-hadrons, with the 77\% efficiency working point. In addition, the small-$R$ jets with $p_T > 30 GeV$ for $2.5 < \left|\eta\right| < 4.5$ are treated as the forward jets. The large-$R$ jets are required to have $p_T > 200 GeV$ and $\left|\eta\right| < 2.0$, containing multiple split jets called the track jets. The flavor identification for the large-$R$ jets depend on the ghost-associated track jets with the MV2c10 discriminant with the 77\% b-tagging efficiency working point. The track jets are depicted more in the following.

The fixed-radius (FR) track jets are applied for $b$-tagging in the previous analysis, using the anti-$k_t$ algorithm with fixed radius parameters. Instead of the FR track jets, the variable-radius (VR) track jets are implemented in the analysis for improving the $b$-tagging efficiency.

\subsection{Leptons}

\subsection{Missing Transverse Momentum}


\section{Event Selection}
hey

\section{Results}
wow

\end{document}